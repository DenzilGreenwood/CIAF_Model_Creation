% Example CIAF Document Using the Styles Package
% This demonstrates how to use the ciaf_document_styles.sty package

\documentclass[12pt,a4paper]{article}

% Load the CIAF styles package
\usepackage{../styles/ciaf_document_styles}

% Setup document metadata
\setupdocument{Sample CIAF Document}{Using the CIAF Styles Package}{Your Name}{v1.0.0}

% Setup header and footer
\setupheaderfooter{Sample Document}{CIAF Styles Demo}

\begin{document}

% Title page
\maketitle
\ciafbigspace

\section{Introduction}

This document demonstrates the use of the CIAF document styles package. All styling, colors, and formatting are automatically applied.

\subsection{Key Features}

\begin{valuebox}
\textbf{Benefits of using the CIAF styles package:}
\begin{itemize}
    \item Consistent styling across all documents
    \item Easy maintenance and updates
    \item Professional CIAF branding
    \item Comprehensive formatting options
\end{itemize}
\end{valuebox}

\subsection{Custom Boxes}

\begin{executivebox}
This is an executive summary box, perfect for highlighting key points and important information.
\end{executivebox}

\ciafvspace

\begin{technicalbox}
This is a technical information box, ideal for technical specifications and detailed explanations.
\end{technicalbox}

\ciafvspace

\begin{riskbox}
This is a risk/warning box for highlighting important warnings, risks, or critical information.
\end{riskbox}

\ciafvspace

\begin{infobox}
This is an information box for general information and notes.
\end{infobox}

\ciafvspace

\begin{alertbox}
This is an alert box for urgent or critical alerts.
\end{alertbox}

\ciafvspace

\begin{contactbox}
This is a contact box for contact information and communication details.
\end{contactbox}

\section{Code Examples}

The package includes professional code formatting:

\begin{lstlisting}[caption={Python Example with CIAF Styling}]
def ciaf_example():
    """Example function demonstrating CIAF code styling"""
    print("Hello, CIAF!")
    return True

# This code is automatically styled with CIAF colors
result = ciaf_example()
\end{lstlisting}

\section{Tables}

Professional table formatting is built-in:

\begin{ciaftable}{lcc}
\textbf{Feature} & \textbf{Traditional} & \textbf{CIAF Styles} \\
\midrule
Consistency & Manual & Automatic \\
Maintenance & Complex & Simple \\
Branding & Variable & Professional \\
\end{ciaftable}

\section{Mathematical Content}

\begin{ciafdefinition}
A CIAF-styled definition looks professional and consistent.
\end{ciafdefinition}

\begin{ciaftheorem}
Mathematical theorems are properly formatted with CIAF colors.
\end{ciaftheorem}

Mathematical equations are also properly styled:
\begin{equation}
\text{CIAF Quality} = \frac{\text{Consistency} \times \text{Professionalism}}{\text{Maintenance Effort}}
\end{equation}

\section{Conclusion}

The CIAF styles package provides a comprehensive solution for consistent document formatting across the entire project.

\end{document}